\documentclass[onecolumn, draftclsnofoot,10pt, compsoc]{IEEEtran}

\usepackage{graphicx}
\usepackage{url}
\usepackage{setspace}
\usepackage{geometry}
\usepackage{listings}
\usepackage{etoolbox}
\usepackage{pdflscape}

\patchcmd{\thebibliography}{\section*{\refname}}{}{}{}

\geometry{textheight=9.5in, textwidth=7in}

% 1. Fill in these details
\def \CapstoneTeamName{			              			 PlanteR-GB}
\def \CapstoneTeamNumber{					           			 Group 64}
\def \GroupMemberOne{				           				Austin Hodgin}
\def \GroupMemberTwo{				           				Travis Hodgin}
\def \GroupMemberThree{			            Maximillian Schmidt}
\def\GroupMemberFour{		        	               Zach Lerew}
\def \CapstoneProjectName{	      	    Winter is Coming...}
\def \CapstoneSponsorCompany{		    Oregon State University}
\def \CapstoneSponsorPerson{		 			  				 Victor Hsu}

% 2. Uncomment the appropriate line below so that the document type works
\def \DocType{		%Problem Statement
				%Requirements Document
				%Technology Review
				Design Document
				%Progress Report
				}

\newcommand{\NameSigPair}[1]{\par
\makebox[2.75in][r]{#1} \hfil 	\makebox[3.25in]{\makebox[2.25in]{\hrulefill} \hfill		\makebox[.75in]{\hrulefill}}
\par\vspace{-12pt} \textit{\tiny\noindent
\makebox[2.75in]{} \hfil		\makebox[3.25in]{\makebox[2.25in][r]{Signature} \hfill	\makebox[.75in][r]{Date}}}}
% 3. If the document is not to be signed, uncomment the RENEWcommand below
\renewcommand{\NameSigPair}[1]{#1}

%%%%%%%%%%%%%%%%%%%%%%%%%%%%%%%%%%%%%%%
\begin{document}
\begin{titlepage}
    \pagenumbering{gobble}
    \begin{singlespace}
    	%\includegraphics[height=4cm]{coe_v_spot1}
        \hfill

        % 4. If you have a logo, use this includegraphics command to put it on the coversheet.
        \includegraphics[height=4cm]{logo.png}

        \par\vspace{.2in}
        \centering
        \scshape{
            \huge CS Capstone \DocType \par
            {\large\today}\par
            \vspace{.5in}
            \textbf{\Huge\CapstoneProjectName}\par

            %\vfill
						\vspace{1in}

            {\large Prepared for}\par
            \Huge \CapstoneSponsorCompany\par
            \vspace{5pt}
            {\Large\NameSigPair{\CapstoneSponsorPerson}\par}

						\vspace{1in}

            {\large Prepared by}\par
						{\huge \CapstoneTeamNumber}\par
            \CapstoneTeamName\par
            \vspace{5pt}

            {
							\Large
							\NameSigPair{\GroupMemberOne}\par
							\NameSigPair{\GroupMemberTwo}\par
							\NameSigPair{\GroupMemberThree}\par
							\NameSigPair{\GroupMemberFour}\par
            }

            \vspace{20pt}
        }
%\textbf{\textsuperscript{citation needed}}
				\newpage
        \begin{abstract}
				\noindent This document details the entire system design for the PlanteR-GB system
        \end{abstract}
    \end{singlespace}
\end{titlepage}

\newpage

\pagenumbering{arabic}
\tableofcontents
% 7. uncomment this (if applicable). Consider adding a page break.
%\listoffigures
%\listoftables
\clearpage
\singlespace

\newpage

% Document outline based on summary of IEEE 1016-2009 from http://www.zeynepaltan.info/SDD-Template.pdf

	% Introduction
	\section{Introduction}
		\subsection{Purpose}
		\subsection{Scope}
		\subsection{Overview}
		\subsection{Definitions and Acronyms}
		\textbf{API} - Application Programming Interface
		\subsection{Reference Material}
			\begingroup
				\renewcommand{\addcontentsline}[3]{}% Remove functionality of \addcontentsline
				\renewcommand{\section}[2]{}% Remove functionality of \section
				%\cite[Sec 3.8]{sourceName}
				\bibliography{ref}
				\bibliographystyle{IEEEtran}
			\endgroup

	\section{System Overview}
		\subsection{Assumptions}
		\subsection{General Constraints}
		\subsection{System Environment}


	\section{System Architecture}
		\subsection{System Diagram}
		NOTE:TO BE REPLACED WITH REAL DIAGRAM

		\includegraphics[width=\linewidth]{sysdiag.png}

		\subsection{Hardware}
			\subsubsection{Overview}
			\subsubsection{Master Controller}
			\subsubsection{Slave Controller}


		\subsection{Control Service}
			\subsubsection{Overview}
			The control service is the bridge between the user interface and the LEDs.
			It has an API that serves as the entry point for all system changes, an internal state representation which is translated to persistent storage, and fed to the LED control system.
			The control service API obfuscates the communication between the client UI and the master controller.
			The API implements a simple method of data communication by opening a series of routes at a specific port on the master controller
			To make a data request or modification, the client sends data to the control service using one of the REST verbs (GET, POST, PUT, DELETE, etc) in addition to its data payload.
			Using the REST protocol simplifies communication between client and server, but also allows new settings and parameters to be quickly changed during development.

			The control service internal state is the primary representation of the settings and configuration of the system.
			This internal state will be the meeting point between the API, data storage, and LED control system.
			It is responsible for representing all data and putting it in persistent storage.
			The system is represented by a selection of objects such as the Zone, Schedule, Controller, and Profile.

			Reading and writing to the state of internal objects changes how the system behaves, and updates the persistent storage using the Data Parser.
			The data parser is responsible for translating the system state into persistent storage.
			In essence, it defines the data contract between the system and storage.

			\subsubsection{API}
			The API will follow the REST protocol. It is an architectural style and modern approach to web service communication. \cite[ch.5]{rest1}
			\subsubsection{Internal State}
			\subsubsection{Data Parser}

		\subsection{Data Design}
			\subsubsection{Overview}
			\subsubsection{Data Description}
			\subsubsection{Zones table}
			\subsubsection{Schedules table}
			\subsubsection{Profiles table}
			\subsubsection{Controllers table}

		\subsection{LED Control System}
			\subsubsection{Overview}
			\subsubsection{Slave Controller Communication}
			\subsubsection{Slave Controller Software}

		\subsection{Stretch goal: Custom Enclosure}
			\subsubsection{Overview}
		\subsection{Stretch goal: External sensors}
			\subsubsection{Overview}

	\section{Human Interface Design}
		\subsection{Overview of User Interface}
		\subsection{Command Line Interface}
			\subsubsection{Interface Overview}
			\subsubsection{Mockup}
		\subsection{Web Interface}
			\subsubsection{Interface Overview}
			\subsubsection{Home Page}
			\subsubsection{Mockup}
			\subsubsection{Zones Page}
			\subsubsection{Mockup}
			%... etc for more pages

	% Scheduling
	\section{Project Timeline}
		\subsection{Iteration 1}
		\subsection{Iteration 2}
		\subsection{Iteration 3}
		\subsection{Iteration 4}
		\subsection{Iteration 5}
		\subsection{Iteration 6}

	%\section{Requirements Matrix} %??? What is this, it was in a template somewhere

	\section{Appendices} % OR glossary??



\end{document}
