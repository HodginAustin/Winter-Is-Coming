\documentclass[onecolumn, draftclsnofoot,10pt, compsoc]{IEEEtran}

\usepackage{float}
\usepackage{graphicx}
\usepackage{url}
\usepackage{setspace}
\usepackage{geometry}
\usepackage{listings}
\usepackage{color}
\usepackage{etoolbox}
\usepackage{pdflscape}

\patchcmd{\thebibliography}{\section*{\refname}}{}{}{}

\geometry{textheight=9.5in, textwidth=7in}

% 1. Fill in these details
\def \CapstoneTeamName{			              			 PlanteR-GB}
\def \CapstoneTeamNumber{					           			 Group 64}
\def \GroupMemberOne{				           				Austin Hodgin}
\def \GroupMemberTwo{				           				Travis Hodgin}
\def \GroupMemberThree{			            Maximillian Schmidt}
\def\GroupMemberFour{		        	               Zach Lerew}
\def \CapstoneProjectName{	      	    Winter is Coming...}
\def \CapstoneSponsorCompany{		    Oregon State University}
\def \CapstoneSponsorPerson{		 			  				 Victor Hsu}

% 2. Uncomment the appropriate line below so that the document type works
\def \DocType{		%Problem Statement
				%Requirements Document
				%Technology Review
				%Design Document
				%Fall Progress Report
				Winter Midterm Progress Report
				}

\newcommand{\NameSigPair}[1]{\par
\makebox[2.75in][r]{#1} \hfil 	\makebox[3.25in]{\makebox[2.25in]{\hrulefill} \hfill		\makebox[.75in]{\hrulefill}}
\par\vspace{-12pt} \textit{\tiny\noindent
\makebox[2.75in]{} \hfil		\makebox[3.25in]{\makebox[2.25in][r]{Signature} \hfill	\makebox[.75in][r]{Date}}}}
% 3. If the document is not to be signed, uncomment the RENEWcommand below
\renewcommand{\NameSigPair}[1]{#1}

%%%%%%%%%%%%%%%%%%%%%%%%%%%%%%%%%%%%%%%
\begin{document}
\begin{titlepage}
    \pagenumbering{gobble}
    \begin{singlespace}
    	%\includegraphics[height=4cm]{coe_v_spot1}
        \hfill

        % 4. If you have a logo, use this includegraphics command to put it on the coversheet.
        \includegraphics[height=4cm]{logo.png}

        \par\vspace{.2in}
        \centering
        \scshape{
            \huge CS Capstone \DocType \par
            {\large\today}\par
            \vspace{.5in}
            \textbf{\Huge\CapstoneProjectName}\par

            %\vfill
						\vspace{1in}

            {\large Prepared for}\par
            \Huge \CapstoneSponsorCompany\par
            \vspace{5pt}
            {\Large\NameSigPair{\CapstoneSponsorPerson}\par}

						\vspace{1in}

            {\large Prepared by}\par
						{\huge \CapstoneTeamNumber}\par
            \CapstoneTeamName\par
            \vspace{5pt}

            {
							\Large
							\NameSigPair{\GroupMemberOne}\par
							\NameSigPair{\GroupMemberTwo}\par
							\NameSigPair{\GroupMemberThree}\par
							\NameSigPair{\GroupMemberFour}\par
            }

            \vspace{20pt}
        }
%\textbf{\textsuperscript{citation needed}}
				\newpage
        \begin{abstract}
				\noindent This document details what the team has accomplished for the second stage of the project.
				It covers the progress made on the programming portion of the project,
				as well as problems we faced and the solutions implemented.
				This document contains four sections written be each of the team's members.
				These sections explain the individual work accomplished by each of us.
        \end{abstract}
    \end{singlespace}
\end{titlepage}

\newpage

\pagenumbering{arabic}
\tableofcontents
% 7. uncomment this (if applicable). Consider adding a page break.
%\listoffigures
%\listoftables
\clearpage
\singlespace

\newpage


% Syntax highlighting
\definecolor{mygreen}{rgb}{0,0.6,0}
\definecolor{mygray}{rgb}{0.5,0.5,0.5}
\definecolor{mymauve}{rgb}{0.58,0,0.82}

\lstset{
  backgroundcolor=\color{white},   % choose the background color
  basicstyle=\footnotesize,        % size of fonts used for the code
  breaklines=true,                 % automatic line breaking only at whitespace
  captionpos=b,                    % sets the caption-position to bottom
  commentstyle=\color{mygreen},    % comment style
  escapeinside={\%*}{*)},          % if you want to add LaTeX within your code
  keywordstyle=\color{blue},       % keyword style
  stringstyle=\color{mymauve},     % string literal style
	frame = single,                  % code framing
}


%%%%%%%%%%%%%%%%%%%%%%% NOTES %%%%%%%%%%%%%%%%%%%%%%%
%
%%%%%%%%%%%%%%%%%%%%%%%%%%%%%%%%%%%%%%%%%%%%%%%%%%%%%


	% Introduction
	\section{Purpose}
	The PlanteR-GB project is a configurable LED strip control system run by microcontrollers.
	The system intends to make growing plants easy and configurable by giving the user control over the state of every LED connected to the system.
	Through a local web interface, the system groups LEDs across multiple light strips into zones.
	Zones set the color, intensity, and power state of their LEDs based on a configurable schedule.
	Existing plant lighting systems offer little to no configuration options and cost between \$70 and \$100. \cite{expensive1} \cite{expensive2} \cite{expensive3}
	Our team is building an LED plant growing system from the ground up to offer configuration options and affordability.
	The trade off made for a lower price is more involvement during setup. Our team intends to release the project as open source software for the DIY community.
	Initially the project will require users to purchase the specific micro controllers used by the system, flash the control software, and wire the system together.
	Our stretch goals include purchasable kits which include all of the necessary parts and instructions to build the system for yourself, and possibly custom enclosures that are sold as pre-made working systems.

	\section{Goals}
	The PlanteR-GB lighting system will allow users to change individual LEDs to a specific color and intensity at any time throughout the day or week.
	LEDs on up to twenty light strips can be grouped into zones and controlled on a schedule with a local web interface.

	\section{Status}
	The project is now in an alpha state. The team's definition of an alpha state is full chain communication.
	This involves sending data from the CLI, through the API, into the internal state, then to the state composer, and finally to the LED control program on the Arduino Nano which lights the LEDs.
	Currently the system handles multiple profiles, zones, daily states, and LED states.
	It does not yet handle multiple controllers and/or LED strips.
	The database currently stores some values, but does not load, update, or delete data.
	These changes are coming before beta, along with bug fixes and improvements.

	\section{Austin Hodgin}
	\section{Travis Hodgin}
	An Arduino Nano is used to control the LEDs. The nano is the device that
	changes the registers attached to the each LED to a specific red, green,
	and blue value. To do this we use the FastLED library, which makes setting
	values fairly painless. We create an LED object based on a static defined
	number of LEDS. The object allows us to change each color value
	per LED.
	\begin{lstlisting}
	#define NUM_LEDS 12 /* adjust to your length of LED */
	#define DATA_PIN 2 /* adjust to the used pin (Arduino nano pin 2 = D2*/
	#define rxPin 0 /* Reciever pin */
	#define txPin 1 /* Transmitter pin */
	CRGB leds[NUM_LEDS]; /*color object */
	\end{lstlisting}
	The Nano is then connected to a Raspberry pi via its RX (receiver)
	and TX (transmitter) pins with a common ground. The connection starts by
	connecting the RX pin of the Pi to the TX pin of the Nano, and the RX pin on
	the Nano to the TX pin on the Pi. Since there is a voltage difference between
	the Pi’s RX and the Nano’s TX we need to use some sort of voltage divider, or
	logic level converter. The Pi uses 3.3V, while the nano uses 5V. To fix this
	we use two resistors, a 1.6KΩ and a 3kΩ to drop the voltage of the nano to
	match the Pi’s, so we don’t fry any components. Once connected, the state composer
	takes care of communication on the pi’s end, as well as getting information to
	pass to the Nano. On the nano, communication is set up by setting up both the
	RX and TX pins depending on the pin number of the specific controller. In this
	case, on the Nano, the TX pin is pin 1, and RX is pin 0. We then set up the pins to be RX to input,
	and TX to output. We then set the baud rate and begin listening.
	\begin{lstlisting}
	void setup() {
  	FastLED.addLeds<NEOPIXEL, DATA_PIN>(leds, NUM_LEDS); /* initialize LEDs */
  	pinMode(rxPin, INPUT);
  	pinMode(txPin, OUTPUT);
  	Serial.begin(57600); /* serial port */
  	FastLED.show();
	}
	\end{lstlisting}
	\\\\
	\noindent Arduino uses a loop function as its main function. Everytime this function
	ends, it will return to the top and start over.  This allows for an easy check.
	During each loop we check the serial buffer for entries. If the serial buffer
	is not empty we then start reading. The first byte contains the LED index used
	to change that specific LED, the second is the red value, third is the green
	value, and finally the fourth has the blue value.  Once we set each value for
	the specified LED index, we call FastLED.show() which will turn the LED on.
	After this we flush the buffer to make sure it no longer contains lingering characters.
 	\\\\
	\noindent Each time we read from the serial buffer we are looking for a character. These
	characters will be typecasted into an integer based on their ascii value.
	This is done as parsing an integer using the native Serial.parseInt() function
	would take to long to set values with the current setup of the state composer.
	ParseInt blocks all other processes until parsing is complete. This was done
	initially, however was moved over to a system using Serial.read() instead
	Serial.read() parses byte by byte rather than reading until a delimiter is
	reached. This allows for faster reading from the serial buffer, which is
	needed for the state composer. Efficiency is a key in the system, we need to
	make sure data from the serial buffer gets read, before its overwritten by the
	following data sent by the state composer.
	\\\\
	\noindent The state composer continuously sends data about the current profile to the
	Nano. This is the main reason efficiency on the Nano is so important. If data
	gets overwritten before the Nano has a chance to grab that, that data may be
	lost. There is, however a failsafe. Because we are continuously sending data
	to the Nano, if data is lost, it will most likely be recovered on the next
	loop. This means that for any given profile, most data loss issues will be
	cleared up fairly quickly, depending on the size of the given zone and profile.
	That being said, our goal is to minimize as many potential data loss situations
	as possible, to make it as seamless as possible for the user.
	\\\\
	\noindent The Nano, also has a limited amount of given space and RAM, which limited our
	potential solutions to the efficiency problem. If we were to take in the entire
	serial buffer as a String, and parse it ourselves, we would run into space
	issues, as Strings take up a lot of space. Reading in a single byte for each
	color and index values is more efficient than parsing either Strings, or
	integers, our testing has found.
	\\\\
	\noindent Code written on the Nano can easily be ported to most types of devices capable
	of running the Arduino coding language. Changes between devices include, LED
	data pin number, RX pin number, TX pin number, as well as changing the
	specific type in the Arduino IDE. Different devices may handle different baud
	rates, however this would need to be changed on the Pi as well, as they need to
	match. Different styles of LEDs different to the NEOPIXEL used in this project
	may also be used, if the FastLED library supports it, by changing the type in
	the initial FastLED setup. All of this means that the arduino code is portable
	to many different hardware styles.
	\\\\
	\noindent In future iterations of the project, as we add more LED controllers, we will
	be implementing a read check system. This involves setting up another pin on
	the Nano, and a separate pin for each controller on the Pi. We need a way to
	communicate between several different controllers at once, however we are
	stuck with only one set of RX and TX pins on the Pi. This means we need to
	“daisy-chain” them together to be able to use the same pins. Because of this
	we have decided to use a pin setup, that will let each individual controller
	know, whether or not the code being sent out is meant for. This may cause a
	problem depending on the number of LED controllers per Pi, as the pi has a
	limited number of usable pins, however this is only a problem for large setups.
	\\\\
	\section{Max Schmidt}
	\section{Zach Lerew}
	For my portion of the project’s ALPHA, I wrote the API, Internal State, and started the Data Parser.
	The API is a RESTful http endpoint which spawns a series of child listeners and defines a set of routes which map to action functions.
	The library I chose to implement this is called Pistache, an element C++ REST framework.
	Each of these routes use a different http verb, depending on the type of CRUD operation the route implements.
	For example, adding a profile to the system uses a POST request because the request introduces new data.
	Adding an existing LED to an existing zone uses a PUT request because it links existing items.
	Viewing information uses the GET verb, and deleting it uses the DELETE verb.
	Each API route maps to an action, much like the ASP.NET framework.
	The action methods take JSON data as input and make changes to the internal state.
	For example, a call to http://localhost:9080/profiles/add with the JSON data
	\begin{lstlisting}
	{“name”:”Test Profile”, “description”: “Winter herbs”}
	\end{lstlisting}
	will create an internal state Profile object with a name and description set.
	The request will return the ID of the newly created object, or an error if the operation was not successful.
	Afterwards, a call to http://localhost:9080/profiles/1 will return the JSON data
	\begin{lstlisting}
	{“name”:”Test Profile”, “description”:”Winter herbs”, “id”:1, “zones”:[]}
	\end{lstlisting}
	When retrieving data, the API uses the Nlohmann JSON library to convert between objects and their text representations. The library has a useful feature that allows me to define from\_json and to\_json functions for each of the internal state objects. These methods are called automatically when implicitly converting the internal state objects into JSON.
	\begin{lstlisting}
	json j_out = *profile;
	\end{lstlisting}
	The conversion works in the opposite direction too, by calling
	\begin{lstlisting}
	json::parse(request.body())
	\end{lstlisting}
	and implicitly casting it to a Profile object. The JSON library also provides numerous methods for easy parsing of JSON objects and lists.
	The API is the entry point for the system as a whole, and is the primary method by which data is entered into the system.
	The command line interface as well as the web interface connect to the API by calling its routes and passing JSON data, so it is essential that the API is well documented and provides all necessary actions.

	\noindent After a request is processed by the API, the internal state takes over by providing a series of object methods which are called in order to add new data or link existing data together.
	The internal state is the in memory representation of the system and all of its settings.
	It has six types of objects. Two of the objects are virtual representations of the physical system, the LED and its associated Controller.
	Two of the objects represent unique permutations of a set of possible states. The first of those is the LEDState, which is a combination of red, green, blue, intensity, and power status.
	The second unique object is the Daily State which represents all led states which occur in a day.
	A single daily state can be used for multiple days of the week, allowing a single daily schedule object to be used every day of the week without creating more data.
	The daily state contains a dictionary of mappings between a time (in seconds after midnight) and pointers to LED states.
	The fourth object is the Zone, which contains a set of 7 daily states and a list of LEDs.
	The Zone has an algorithm in it which looks through the set of 7 daily states and finds a pointer to the current LED state which should be active based on the system's day and time.
	If an LED state cannot be found anywhere in the week, the zero LED state is returned, which represents an state of no color, no intensity, and no power.
	The final object is the Profile, which contains a name, description, and set of zones.
	Each profile represents an overall state of the system, a unique combination of a schedule and a set of LEDs.
	One possible user story would be a series of four profiles, one for each season.
	During different seasons the user may grow different vegetables depending on seed availability and preference. Changes made to the internal state exist only in memory, and are lost when the system resets, or in the case of our alpha, crashes.

	\noindent The data parser exists to save the data within the internal state into the file system and later retrieve it back into memory.
	The chosen library is sqlite\_orm by fnc12 on github.
	This library is likely the most flexible database Object Relational Mapping system available for C++ while still being quick to learn.
	This system allows you to build the database schema using objects by passing the results of a make\_table call to a make\_storage function.
	The make\_table function takes a series of make\_column calls which in turn take pointers to accessor and mutator functions for each of the objects in the system.
	Once the storage object is created on the file system and the schema defined in memory, the ORM provides a series of SQL like functions such as insert, update, replace, remove, and so on.
	These functions take an instance of an object such as a Zone, and automatically generate and call the appropriate SQL code.
	The ORM also provides lamba style functions which replicate SQL functions like JOIN, WHICH, and aggregate functions such as COUNT or SUM.
	In a similar manner as the JSON library, the sqlite\_orm library takes a heavy amount of setup for schema definition, after which subsequent operations are quick and painless.
	Once the data is stored on the file system, it can be loaded in the dataParser’s initialize function.
	The ORM library provides a series of select functions which allow a direct translation into standard C++ STL container containing objects.
	My favorite part of this ORM library is the insert functions which take an internal state object and returns the ID of the newly created object.
	This allows me to keep object IDs unique easily as well as pass them around in the API as objects are created, modified, and deleted.
	During the project so far, I have been responsible for a lot of the middle parts of the system, and as such have been responsible for making sure my team understands what data to give me, and what data they are getting from me.



	\section{Problems}
		\subsection{C++ is a terrible language for web development}
			C++ is a language not intended to be used for serious web development or API writing.
			Therefore, many libraries are small, difficult to use, and have few contributors.
			This creates a more difficult learning curve, but was ultimately necessary due to hardware communication.
		\subsubsection{Solution}
			We spent extra time doing research into libraries and solutions in order to give us the best chance of success given the circumstances. We also chose to write the CLI in Python to speed up development.
	\subsection{Libraries Not well documented}
      Many open source or non-supported libraries include limited or non existent documentation.
			This can be a hindrance when developing and issues come up, as support is limited.
  	\subsubsection{Solution}
	    In a few cases we were able to communicate with the library developer and get in depth answers,
			otherwise we were forced to look through source code and make educated guesses.
	\subsection{Database many-to-many relationships}
      The database ORM chosen does not have an included method of supporting many-to-many relationships.
			The system has two of these relationships.
  	\subsubsection{Solution}
	    We were able to post a question to the development GitHub and receive a response from the developer with a suggestion on how to solve the problem. A set of three temporary relationship objects were created, which only exist when saving and loading data.
	 	\subsection{Five Person Scheduling and communication}
		 Seeing that the group consists of four members, and our client's schedule contains busy times similar to ours, scheduling times that worked for everyone
		 forced compromises in meeting times.
  	\subsubsection{Solution}
		  With no "perfect" times aligning in anyone's schedules, some group members had to cut time from work, other class work, or personal time to work on the project.
		  There was no "clean" solution to the scheduling issue.


	\section{Remaining work}
	Data parser
	CLI Multi zones, colors
	Web interface
	State composer multiple microcontrollers
	Large system tests


	\section{References}
			\begingroup
				\renewcommand{\addcontentsline}[3]{}% Remove functionality of \addcontentsline
				\renewcommand{\section}[2]{}% Remove functionality of \section
				%\cite[Sec 3.8]{sourceName}
				\bibliography{ref}
				\bibliographystyle{IEEEtran}
			\endgroup
\end{document}
