\documentclass[onecolumn, draftclsnofoot,10pt, compsoc]{IEEEtran}

\usepackage{float}
\usepackage{graphicx}
\usepackage{url}
\usepackage{setspace}
\usepackage{geometry}
\usepackage{listings}
\usepackage{color}
\usepackage{etoolbox}
\usepackage{pdflscape}

\patchcmd{\thebibliography}{\section*{\refname}}{}{}{}

\geometry{textheight=9.5in, textwidth=7in}

% 1. Fill in these details
\def \CapstoneTeamName{			              			 PlanteR-GB}
\def \CapstoneTeamNumber{					           			 Group 64}
\def \GroupMemberOne{				           				Austin Hodgin}
\def \GroupMemberTwo{				           				Travis Hodgin}
\def \GroupMemberThree{			            Maximillian Schmidt}
\def\GroupMemberFour{		        	               Zach Lerew}
\def \CapstoneProjectName{	      	    Winter is Coming...}
\def \CapstoneSponsorCompany{		    Oregon State University}
\def \CapstoneSponsorPerson{		 			  				 Victor Hsu}

% 2. Uncomment the appropriate line below so that the document type works
\def \DocType{		%Problem Statement
				%Requirements Document
				%Technology Review
				%Design Document
				Fall Progress Report
				}

\newcommand{\NameSigPair}[1]{\par
\makebox[2.75in][r]{#1} \hfil 	\makebox[3.25in]{\makebox[2.25in]{\hrulefill} \hfill		\makebox[.75in]{\hrulefill}}
\par\vspace{-12pt} \textit{\tiny\noindent
\makebox[2.75in]{} \hfil		\makebox[3.25in]{\makebox[2.25in][r]{Signature} \hfill	\makebox[.75in][r]{Date}}}}
% 3. If the document is not to be signed, uncomment the RENEWcommand below
\renewcommand{\NameSigPair}[1]{#1}

%%%%%%%%%%%%%%%%%%%%%%%%%%%%%%%%%%%%%%%
\begin{document}
\begin{titlepage}
    \pagenumbering{gobble}
    \begin{singlespace}
    	%\includegraphics[height=4cm]{coe_v_spot1}
        \hfill

        % 4. If you have a logo, use this includegraphics command to put it on the coversheet.
        \includegraphics[height=4cm]{logo.png}

        \par\vspace{.2in}
        \centering
        \scshape{
            \huge CS Capstone \DocType \par
            {\large\today}\par
            \vspace{.5in}
            \textbf{\Huge\CapstoneProjectName}\par

            %\vfill
						\vspace{1in}

            {\large Prepared for}\par
            \Huge \CapstoneSponsorCompany\par
            \vspace{5pt}
            {\Large\NameSigPair{\CapstoneSponsorPerson}\par}

						\vspace{1in}

            {\large Prepared by}\par
						{\huge \CapstoneTeamNumber}\par
            \CapstoneTeamName\par
            \vspace{5pt}

            {
							\Large
							\NameSigPair{\GroupMemberOne}\par
							\NameSigPair{\GroupMemberTwo}\par
							\NameSigPair{\GroupMemberThree}\par
							\NameSigPair{\GroupMemberFour}\par
            }

            \vspace{20pt}
        }
%\textbf{\textsuperscript{citation needed}}
				\newpage
        \begin{abstract}
				\noindent This document details
        \end{abstract}
    \end{singlespace}
\end{titlepage}

\newpage

\pagenumbering{arabic}
\tableofcontents
% 7. uncomment this (if applicable). Consider adding a page break.
%\listoffigures
%\listoftables
\clearpage
\singlespace

\newpage


% Syntax highlighting
\definecolor{mygreen}{rgb}{0,0.6,0}
\definecolor{mygray}{rgb}{0.5,0.5,0.5}
\definecolor{mymauve}{rgb}{0.58,0,0.82}

\lstset{
  backgroundcolor=\color{white},   % choose the background color
  basicstyle=\footnotesize,        % size of fonts used for the code
  breaklines=true,                 % automatic line breaking only at whitespace
  captionpos=b,                    % sets the caption-position to bottom
  commentstyle=\color{mygreen},    % comment style
  escapeinside={\%*}{*)},          % if you want to add LaTeX within your code
  keywordstyle=\color{blue},       % keyword style
  stringstyle=\color{mymauve},     % string literal style
	frame = single,                  % code framing
}


%%%%%%%%%%%%%%%%%%%%%%% NOTES %%%%%%%%%%%%%%%%%%%%%%%
%
%%%%%%%%%%%%%%%%%%%%%%%%%%%%%%%%%%%%%%%%%%%%%%%%%%%%%


	% Introduction
	\section{Purpose}
	The PlanteR-GB project is a configurable LED strip control system run by microcontrollers.
	The system intends to make growing plants easy and configurable by giving the user control over the state of every LED connected to the system.
	Through a local web interface, the system groups LEDs across multiple light strips into zones.
	Zones set the color, intensity, and power state of their LEDs based on a configurable schedule.
	Existing plant lighting systems offer little to no configuration options and cost between \$70 and \$100. \cite{expensive1} \cite{expensive2} \cite{expensive3}
	Our team is building an LED plant growing system from the ground up to offer configuration options and affordability.
	The trade off made for a lower price is more involvement during setup. Our team intends to release the project as open source software for the DIY community.
	Initially the project will require users to purchase the specific micro controllers used by the system, flash the control software, and wire the system together.
	Our stretch goals include purchasable kits which include all of the necessary parts and instructions to build the system for yourself, and possibly custom enclosures that are sold as pre-made working systems.

	\section{Goals}
	The PlanteR-GB lighting system will allow users to change individual LEDs to a specific color and intensity at any time throughout the day or week.
	LEDs on up to twenty light strips can be grouped into zones and controlled on a schedule with a local web interface.

	\section{Status}
	The current status of the project is...
		\subsection{Week 1}
		The first week of capstone primarily involved learning about the structure of the class, what is going to be expected of us for the following year, and an introduction to the projects that we chose from.
		Each of us picked our top choices out of almost 80 projects. We all shared a top choice of the \textit{Winter Is Coming...} project by Victor Hsu.
		Before being chosen for the project, each of us wrote descriptions for our top five projects.

		\subsection{Week 2}
		Week two was all about establishing communication. The beginning of the capstone class was full of uncertainty and chaos.
		Once teams were chosen and announced, we made contact and got to know each other. We created a slack channel for instant messaging between us, and figured out our combined schedules.
		With an established set of times when we were all available, we began formulating an email to our client. We set up a meeting time and began writing down our questions and ideas.

		\noindent \\We met with our client, Victor Hsu, and discussed the expectations he had, the purpose of the project, and the ideas we had.
		Our client had a small set of deliverables, but by and large gave us freedom over the project.
		Later in the week, our team met up to sketch out all of the features we wanted and how we could design a system that met both our requirements, and the client's requirements.

		\subsection{Week 3}
		Week three formed the ideas and concept images our team had created after meeting our client into a coherent statement of the problem and reason for why our project was necessary.
		We quickly created a rough draft of the problem statement document, and then met with our TA Junki Hong to discuss his role in our project.
		Our TA is a source of knowledge and information for general details, but specifics may still be passed along to our instructors for clarification.

		\subsection{Week 4}
		\subsection{Week 5}
		\subsection{Week 6}
		\subsection{Week 7}
		\subsection{Week 8}
		\subsection{Week 9}
		\subsection{Week 10}
		\subsection{Week 11}

	\section{Problems}
		\subsection{Blah blah}
			\subsubsection{Solution}
		\subsection{Foo Bar}
			\subsubsection{Solution}

	\section{Retrospective}
	\begin{tabular}{ |c|c|c|c|c| }
		\hline
		Positives & Deltas & Actions \\
		\hline
		A & B & C \\
		\hline
	\end{tabular}
\end{document}
