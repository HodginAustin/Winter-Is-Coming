\documentclass[onecolumn, draftclsnofoot,10pt, compsoc]{IEEEtran}

\usepackage{float}
\usepackage{graphicx}
\usepackage{url}
\usepackage{setspace}
\usepackage{geometry}
\usepackage{listings}
\usepackage{color}
\usepackage{etoolbox}
\usepackage{pdflscape}

\patchcmd{\thebibliography}{\section*{\refname}}{}{}{}

\geometry{textheight=9.5in, textwidth=7in}

% 1. Fill in these details
\def \CapstoneTeamName{			              			 PlanteR-GB}
\def \CapstoneTeamNumber{					           			 Group 64}
\def \GroupMemberOne{				           				Austin Hodgin}
\def \GroupMemberTwo{				           				Travis Hodgin}
\def \GroupMemberThree{			            Maximillian Schmidt}
\def\GroupMemberFour{		        	               Zach Lerew}
\def \CapstoneProjectName{	      	    Winter is Coming...}
\def \CapstoneSponsorCompany{		    Oregon State University}
\def \CapstoneSponsorPerson{		 			  				 Victor Hsu}

% 2. Uncomment the appropriate line below so that the document type works
\def \DocType{		%Problem Statement
				%Requirements Document
				%Technology Review
				%Design Document
				Fall Progress Report
				}

\newcommand{\NameSigPair}[1]{\par
\makebox[2.75in][r]{#1} \hfil 	\makebox[3.25in]{\makebox[2.25in]{\hrulefill} \hfill		\makebox[.75in]{\hrulefill}}
\par\vspace{-12pt} \textit{\tiny\noindent
\makebox[2.75in]{} \hfil		\makebox[3.25in]{\makebox[2.25in][r]{Signature} \hfill	\makebox[.75in][r]{Date}}}}
% 3. If the document is not to be signed, uncomment the RENEWcommand below
\renewcommand{\NameSigPair}[1]{#1}

%%%%%%%%%%%%%%%%%%%%%%%%%%%%%%%%%%%%%%%
\begin{document}
\begin{titlepage}
    \pagenumbering{gobble}
    \begin{singlespace}
    	%\includegraphics[height=4cm]{coe_v_spot1}
        \hfill

        % 4. If you have a logo, use this includegraphics command to put it on the coversheet.
        \includegraphics[height=4cm]{logo.png}

        \par\vspace{.2in}
        \centering
        \scshape{
            \huge CS Capstone \DocType \par
            {\large\today}\par
            \vspace{.5in}
            \textbf{\Huge\CapstoneProjectName}\par

            %\vfill
						\vspace{1in}

            {\large Prepared for}\par
            \Huge \CapstoneSponsorCompany\par
            \vspace{5pt}
            {\Large\NameSigPair{\CapstoneSponsorPerson}\par}

						\vspace{1in}

            {\large Prepared by}\par
						{\huge \CapstoneTeamNumber}\par
            \CapstoneTeamName\par
            \vspace{5pt}

            {
							\Large
							\NameSigPair{\GroupMemberOne}\par
							\NameSigPair{\GroupMemberTwo}\par
							\NameSigPair{\GroupMemberThree}\par
							\NameSigPair{\GroupMemberFour}\par
            }

            \vspace{20pt}
        }
%\textbf{\textsuperscript{citation needed}}
				\newpage
        \begin{abstract}
				\noindent This document details what the group has accomplished for the first stage of the project.
				It covers each of the documents created over the course of the term.  It also specifically details,
				week by week, what work the team completed, and what was communicated to the client as well.
				Lastly, the document mentions the postives, negatives, and other thoughts about the current stage
				and progress of the project.  
        \end{abstract}
    \end{singlespace}
\end{titlepage}

\newpage

\pagenumbering{arabic}
\tableofcontents
% 7. uncomment this (if applicable). Consider adding a page break.
%\listoffigures
%\listoftables
\clearpage
\singlespace

\newpage


% Syntax highlighting
\definecolor{mygreen}{rgb}{0,0.6,0}
\definecolor{mygray}{rgb}{0.5,0.5,0.5}
\definecolor{mymauve}{rgb}{0.58,0,0.82}

\lstset{
  backgroundcolor=\color{white},   % choose the background color
  basicstyle=\footnotesize,        % size of fonts used for the code
  breaklines=true,                 % automatic line breaking only at whitespace
  captionpos=b,                    % sets the caption-position to bottom
  commentstyle=\color{mygreen},    % comment style
  escapeinside={\%*}{*)},          % if you want to add LaTeX within your code
  keywordstyle=\color{blue},       % keyword style
  stringstyle=\color{mymauve},     % string literal style
	frame = single,                  % code framing
}


%%%%%%%%%%%%%%%%%%%%%%% NOTES %%%%%%%%%%%%%%%%%%%%%%%
%
%%%%%%%%%%%%%%%%%%%%%%%%%%%%%%%%%%%%%%%%%%%%%%%%%%%%%


	% Introduction
	\section{Purpose}
	The PlanteR-GB project is a configurable LED strip control system run by microcontrollers.
	The system intends to make growing plants easy and configurable by giving the user control over the state of every LED connected to the system.
	Through a local web interface, the system groups LEDs across multiple light strips into zones.
	Zones set the color, intensity, and power state of their LEDs based on a configurable schedule.
	Existing plant lighting systems offer little to no configuration options and cost between \$70 and \$100. \cite{expensive1} \cite{expensive2} \cite{expensive3}
	Our team is building an LED plant growing system from the ground up to offer configuration options and affordability.
	The trade off made for a lower price is more involvement during setup. Our team intends to release the project as open source software for the DIY community.
	Initially the project will require users to purchase the specific micro controllers used by the system, flash the control software, and wire the system together.
	Our stretch goals include purchasable kits which include all of the necessary parts and instructions to build the system for yourself, and possibly custom enclosures that are sold as pre-made working systems.

	\section{Goals}
	The PlanteR-GB lighting system will allow users to change individual LEDs to a specific color and intensity at any time throughout the day or week.
	LEDs on up to twenty light strips can be grouped into zones and controlled on a schedule with a local web interface.

	\section{Status}
	The current status of the project remains in the designing and planning stage.  The team has created four official documents detailing the project's problem statement, the project requirements,
	a technology review for the system components, and a design document explaining how each of technologies are implemented. Between terms over the break, some of the team members will be prototyping
	the more complex components.

		\subsection{Week 1}
		The first week of capstone primarily involved learning about the structure of the class, what is going to be expected of us for the following year, and an introduction to the projects that we chose from.
		Each of us picked our top choices out of almost 80 projects. We all shared a top choice of the \textit{Winter Is Coming...} project by Victor Hsu.
		Before being chosen for the project, each of us wrote descriptions for our top five projects.

		\subsection{Week 2}
		Week two was all about establishing communication. The beginning of the capstone class was full of uncertainty and chaos.
		Once teams were chosen and announced, we made contact and got to know each other. We created a slack channel for instant messaging between us, and figured out our combined schedules.
		With an established set of times when we were all available, we began formulating an email to our client. We set up a meeting time and began writing down our questions and ideas.

		\noindent \\We met with our client, Victor Hsu, and discussed the expectations he had, the purpose of the project, and the ideas we had.
		Our client had a small set of deliverables, but by and large gave us freedom over the project.
		Later in the week, our team met up to sketch out all of the features we wanted and how we could design a system that met both our requirements, and the client's requirements.

		\subsection{Week 3}
		Week three formed the ideas and concept images our team had created after meeting our client into a coherent statement of the problem and reason for why our project was necessary.
		We quickly created a rough draft of the problem statement document, and then met with our TA, Junki Hong, to discuss his role in our project.
		Our TA is a source of knowledge and information for general details, but specifics may still be passed along to our instructors for clarification.

		\subsection{Week 4}
		Week four consisted of finishing the problem statement. After finishing the final draft we emailed our client a copy for approval and started work on the requirements document.
		We promptly started work on our requirements document. We met several as a group times during the week to discuss what the requirements for the project  to be considered a success.
		Initial research for some of the primary hardware was started. We met with Junki Hong to ensure we are on track.
		Finally we organized our github, adding a documents folder, and setting the problem statement to be our READme.

		\subsection{Week 5}
		The rough draft of our requirements document was due Week 5. After several meetings we made a few decisions on how the project will be built, such as what some of the hardware we will be using.
		Once our rough draft was completed we met with our client to go over our requirements document.
		Further research on potential hardware and software was made in an effort to fully understand the extent of the requirements as well as time needed for each iteration.
		We met with Junki Hong to make sure we were on track with our requirements document.  We also had a meeting with Victor to discuss the rough draft of the requirements document, the upcoming technology
		reviews, and the design document.

		\subsection{Week 6}
		With week six the final draft of our requirements document was due. The technology review was assigned.
		Combined with our previous research, we continued our research. We distributed three parts to each member and started writing.
		During our meeting with Junki we discussed our requirements document and our issues getting the table of contents workings.

		\subsection{Week 7}
		We continued working on the technology review rough draft which was due the following week.
		After our first meeting for the week, we created a outline for our individual technology reviews.
		Our weekly meeting with Junki consisted of talking about our requirements document.
		We discussed our technology reviews, as we were a little confused when they will be merged together.
		We also heavily worked on the over all system design, getting into system component concepts, and the necessary data flow.

		\subsection{Week 8}
		The final draft of our individual technology review was due at the end of this week. Most of the technology we eventually reviewed helped define design flow.
		We received peer review for the tech reviews during class and promptly started editing.
		The team met to talk about our technology reviews, and started on our design document.

		\subsection{Week 9}
		With Thanksgiving, this was a short week. The final draft of the technology review was due. We each emailed Kirsten our reviews.
		We did not have any class meetings, nor a meeting with Junki.
		We officially started working on our design document due the following week.

		\subsection{Week 10}
		We completed our design document this week. After finishing our rough draft we gave a copy along with a copy of each of our technology reviews to our client.
		Client verification is not due until next term, however providing these documents early gives us more time to implement any feedback the client might have.

		\section{Problems}
			\subsection{Latex Table Of Contents}
			During week 6 we experienced a problem where we could not get our table
			of contents to generate in our latex documents.
				\subsubsection{Solution}
				The Solution to this was we were using the incorrect formatting for each
				section. Instead of using the \verb|\section*| syntax to generate a section that
				didn't have a number we started using \verb|\section| This allowed the table
				to be generated.
			\subsection{Libraries Not On Desired Platform}
      Originally we were wanting to use the raspberry pi for our controllers. After
      looking at all the libraries needed to run the LEDs we found out that the
      pi would not be able to use them due to the fact that the libraries are very
      dependent on clock cycle and the OS can interrupt it.
		  	\subsubsection{Solution}
		    The solution that we came up with to solve this is to use a raspberry pi to
		    be the main controller and another microcontroller would be the device that
		    would be controlling the LEDs.
		 \subsection{Five Person Scheduling}
		 Seeing that the group consists of four members, and our client's schedule contains busy times similar to ours, scheduling times that worked for everyone
		 forced compromises in meeting times.
		 	  \subsubsection{Solution}
			  With no "perfect" times aligning in anyone's schedules, some group members had to cut time from work, other class work, or personal time to work on the project.
			  There was no "clean" solution to the scheduling issue.


	\section{Retrospective}
	\begin{tabular}{ |c|c|c|c|c| }
		\hline
		Positives & Deltas & Actions \\
		\hline
		Group Interaction & Scheduling with 5 People & Develop Schedules Around Preset Group Work Times \\
		Solid Understanding of LaTeX & Learning Around Latex & Keep Using Latex and Familiarize More \\
		Personal Problem and Solution & Uncertainties Around Due Dates & Continue Persistent Communications With Instructors \\
		Professional Documentation & Getting Ahead of Ourselves & Take a Step Back per Process Until Complete \\
		Working with Non-CS Client & & \\
		Flexible Deadlines & &  \\
		\hline
	\end{tabular}

	\section{References}
			\begingroup
				\renewcommand{\addcontentsline}[3]{}% Remove functionality of \addcontentsline
				\renewcommand{\section}[2]{}% Remove functionality of \section
				%\cite[Sec 3.8]{sourceName}
				\bibliography{ref}
				\bibliographystyle{IEEEtran}
			\endgroup
\end{document}
