\documentclass[onecolumn, draftclsnofoot,10pt, compsoc]{IEEEtran}

\usepackage{graphicx}
\usepackage{url}
\usepackage{setspace}
\usepackage{geometry}
\usepackage{listings}

\geometry{textheight=9.5in, textwidth=7in}

% 1. Fill in these details
\def \CapstoneTeamName{			              			 PlanteR-GB}
\def \CapstoneTeamNumber{					           			 Group 64}
\def \GroupMemberOne{				           				Austin Hodgin}
\def \GroupMemberTwo{				           				Travis Hodgin}
\def \GroupMemberThree{			            Maximillian Schmidt}
\def\GroupMemberFour{		        	               Zach Lerew}
\def \CapstoneProjectName{	      	    Winter is Coming...}
\def \CapstoneSponsorCompany{		    Oregon State University}
\def \CapstoneSponsorPerson{		 			  				 Victor Hsu}

% 2. Uncomment the appropriate line below so that the document type works
\def \DocType{		%Problem Statement
				Requirements Document
				%Technology Review
				%Design Document
				%Progress Report
				}

\newcommand{\NameSigPair}[1]{\par
\makebox[2.75in][r]{#1} \hfil 	\makebox[3.25in]{\makebox[2.25in]{\hrulefill} \hfill		\makebox[.75in]{\hrulefill}}
\par\vspace{-12pt} \textit{\tiny\noindent
\makebox[2.75in]{} \hfil		\makebox[3.25in]{\makebox[2.25in][r]{Signature} \hfill	\makebox[.75in][r]{Date}}}}
% 3. If the document is not to be signed, uncomment the RENEWcommand below
\renewcommand{\NameSigPair}[1]{#1}

%%%%%%%%%%%%%%%%%%%%%%%%%%%%%%%%%%%%%%%
\begin{document}
\begin{titlepage}
    \pagenumbering{gobble}
    \begin{singlespace}
    	%\includegraphics[height=4cm]{coe_v_spot1}
        \hfill

        % 4. If you have a logo, use this includegraphics command to put it on the coversheet.
        \includegraphics[height=4cm]{derp.jpg}

        \par\vspace{.2in}
        \centering
        \scshape{
            \huge CS Capstone \DocType \par
            {\large\today}\par
            \vspace{.5in}
            \textbf{\Huge\CapstoneProjectName}\par

            %\vfill
						\vspace{1in}

            {\large Prepared for}\par
            \Huge \CapstoneSponsorCompany\par
            \vspace{5pt}
            {\Large\NameSigPair{\CapstoneSponsorPerson}\par}

						\vspace{1in}

            {\large Prepared by}\par
						{\huge \CapstoneTeamNumber}\par
            \CapstoneTeamName\par
            \vspace{5pt}

            {
							\Large
							\NameSigPair{\GroupMemberOne}\par
							\NameSigPair{\GroupMemberTwo}\par
							\NameSigPair{\GroupMemberThree}\par
							\NameSigPair{\GroupMemberFour}\par
            }

            \vspace{20pt}
        }

				\newpage
        \begin{abstract}
				\noindent This project will create a system that controls RGB LEDs responsible for indoor plant growth during the cold seasons.
				Seasonal weather conditions may vary between different parts of the world, but plant growth becomes especially difficult during the fall and winter months.
				Oregon is an exceptional case of this rule. The state's winter season produces a hostile growth environment for plants such as herbs, spices, and decorative flowers.
				Plants like these are not able to survive the cold, dark, and humid Oregon weather.

				\noindent Bringing the plants into a more friendly and manageable indoor environment can still prove to be a difficult task.
				Plants need specific conditions for soil, water, temperature, and light.
				Existing indoor lighting systems can be difficult to use, expensive, and provide few customization options.
				Research provided by the client \textbf{\textsuperscript{citation needed}} has shown that some plants grow differently under different colors(wavelengths) of light.
				This project aims to produce a system that can control the color, intensity, and schedule for multiple zones of RGB LEDs.
				The system produced by these efforts will have a simple user interface through which all control settings can be viewed and manipulated.

				\noindent Once given control over the growing conditions, the plants will have a better chance of successful growth, and at the same time the impact on the user's busy life will be reduced.
				The system will be driven by a microprocessor that will manipulate the color, intensity, and schedule for multiple zones of RGB LEDs.
				Along side this hardware, the development of an intuitive user interface will allow the user to control the system with minimal physical interaction.
				The project has a simple core, but many stretch goals have been designed to increase functionality and end usability should time allow.

        \end{abstract}
    \end{singlespace}
\end{titlepage}

\newpage

\pagenumbering{arabic}
\tableofcontents
% 7. uncomment this (if applicable). Consider adding a page break.
%\listoffigures
%\listoftables
\clearpage
\singlespace


	% Document body
	\section*{Introduction}

		\subsection*{Purpose}

		\subsection*{Scope}

		\subsection*{Definitions, Acronyms, and Abbreviations}

		% References
		\subsection*{References}
		%\cite[Sec 3.8]{sourceName}
		%\bibliographystyle{IEEEtran}
		%\bibliography{./ref}


		\subsection*{Overview}


	\section*{Overall Description}
		\subsection*{Product Perspective}

		\subsection*{Product Functions}

		\subsection*{User Characteristics}

		\subsection*{Constraints}

		\subsection*{Assumptions and Dependencies}

	\section*{Specific Requirements}
		\subsection*{Required features - v1.0}

		\subsection*{Additional features - v2.0}

		\subsection*{Stretch goals - v3.0}

		\section*{Performance Metrics}


\end{document}
